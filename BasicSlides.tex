\documentclass[xcolor=dvipsnames]{beamer} 
%% Add [xcolor=dvipsnames] to use beamer colors

%------------------------------------

%% Boadilla theme is used 
\mode<presentation> {
\usetheme{Boadilla}
\setbeamertemplate{navigation symbols}{}
\setbeamertemplate{footline}[frame number] %% For frame numbering
}

%------------------------------------

%% For using footnote w/o numbers 
\makeatletter
\def\blfootnote{\gdef\@thefnmark{}\@footnotetext}
\makeatother

%------------------------------------

%% Add your favourite packages below 
\usepackage{tikz} % Schematics
\usepackage{xcolor} % Color
\usepackage{amsmath} % Equations
\usepackage{braket} % BraKet notation
\usepackage{bm} % Bold equation
\usepackage{graphicx} % Images

%------------------------------------

%% Title
\title{\Large{\textbf{Title of your presentation}}}

%% Default title color is blue. To change color:
\setbeamercolor{title}{fg=Black, bg=Blue!20}
%% fg = foreground, bg = background, ! = opacity

%% Author
\author{\large{Your Name}}

%% Date (chosen today). Can be changed to a custom date
\date{\today}

%------------------------------------

%% Start the document
\begin{document}

%% Title page
\begin{frame}
\titlepage % Print the title page as the first slide
\end{frame}

%------------------------------------

%% Default frametitle color is blue. To change color:
\setbeamercolor{frametitle}{fg=Black, bg=Blue!20}

%------------------------------------
%% Start the frames/slides %%
%------------------------------------


%% Use section (and subsections) to organize
\section{Section Name}

\begin{frame}
\frametitle{Frame title}

We can add words, equations, images, etc.

%% Add an image
\centering
%\includegraphics[width=0.7\linewidth]{FilePath/PictureFile.format}

\end{frame}

%----------------------------------------

\begin{frame}
\frametitle{Frame with bullet points}

\begin{itemize}
\setlength \itemsep{1.5 em} % Space between bullets

\item<2-> This bullet appears\footnotemark[1]

\item<3-> And then this

\item<4-> And so on

\item<5-> If a sentence is too long, we can choose 
\\ where to break our line % Using \\

\end{itemize}

\footnotetext[1]{\tiny{Some footnote text}}
\end{frame}

%----------------------------------------

\begin{frame}
\frametitle{Frames vs slides}

\only<1>{
Usually frame and slide are interchangeable.
}
\only<2>{
\center{But we can also have multiple slides in \textit{one} frame.}
}
\only<3>{
\center{We can \textbf{highlight} using \alert{red} or a {\color{RoyalPurple} specific} color}
}
\only<4>{
\center{We can also add color to the equations}

$$ \color{RoyalPurple} 
H \ket{\Psi} = E \ket{\Psi}
$$
}
\end{frame}


\end{document}
